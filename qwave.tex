\documentclass[epsfig,12pt]{article}
\usepackage{epsfig}
\usepackage{graphicx}
\usepackage{rotating}
\usepackage{latexsym}
\usepackage{amsmath}
\usepackage{amssymb}
\usepackage{relsize}
\usepackage{geometry}
\geometry{letterpaper}
\usepackage{color}
\usepackage{bm}
\usepackage{slashed}
\usepackage{showlabels}




%%%%%%%%%%%%%%%%%%%%%%%%%%%%%%%%%%%%%%%%%%%%%%%%%%%%%%%%%%%%%%%%%%%%%%%%%%%%%%%%
%                                                                              %
%                                                                              %
%                     D O C U M E N T   S E T T I N G S                        %
%                                                                              %
%                                                                              %
%%%%%%%%%%%%%%%%%%%%%%%%%%%%%%%%%%%%%%%%%%%%%%%%%%%%%%%%%%%%%%%%%%%%%%%%%%%%%%%%
\def\baselinestretch{1.1}
\renewcommand{\theequation}{\thesection.\arabic{equation}}

\hyphenation{con-fi-ning}
\hyphenation{Cou-lomb}
\hyphenation{Yan-ki-e-lo-wicz}




%%%%%%%%%%%%%%%%%%%%%%%%%%%%%%%%%%%%%%%%%%%%%%%%%%%%%%%%%%%%%%%%%%%%%%%%%%%%%%%%
%                                                                              %
%                                                                              %
%                      C O M M O N   D E F I N I T I O N S                     %
%                                                                              %
%                                                                              %
%%%%%%%%%%%%%%%%%%%%%%%%%%%%%%%%%%%%%%%%%%%%%%%%%%%%%%%%%%%%%%%%%%%%%%%%%%%%%%%%
\def\beq{\begin{equation}}
\def\eeq{\end{equation}}
\def\beqn{\begin{eqnarray}}
\def\eeqn{\end{eqnarray}}
\def\beqn{\begin{eqnarray}}
\def\eeqn{\end{eqnarray}}
\def\nn{\nonumber}
\def\ba{\beq\new\begin{array}{c}}
\def\ea{\end{array}\eeq}
\def\be{\ba}
\def\ee{\ea}


\newcommand{\nfour}{${\cal N}=4\;$}
\newcommand{\none}{${\mathcal N}=1\,$}
\newcommand{\nonen}{${\mathcal N}=1$}
\newcommand{\ntwo}{${\mathcal N}=2$}
\newcommand{\ntt}{${\mathcal N}=(2,2)\,$}
\newcommand{\nzt}{${\mathcal N}=(0,2)\,$}
\newcommand{\ntwon}{${\mathcal N}=2$}
\newcommand{\ntwot}{${\mathcal N}= \left(2,2\right) $ }
\newcommand{\ntwoo}{${\mathcal N}= \left(0,2\right) $ }
\newcommand{\ntwoon}{${\mathcal N}= \left(0,2\right)$}


\newcommand{\ca}{{\mathcal A}}
\newcommand{\cell}{{\mathcal L}}
\newcommand{\cw}{{\mathcal W}}
\newcommand{\cs}{{\mathcal S}}
\newcommand{\vp}{\varphi}
\newcommand{\pt}{\partial}
\newcommand{\ve}{\varepsilon}
\newcommand{\gs}{g^{2}}
\newcommand{\zn}{$Z_N$}
\newcommand{\cd}{${\mathcal D}$}
\newcommand{\cde}{{\mathcal D}}
\newcommand{\cf}{${\mathcal F}$}
\newcommand{\cfe}{{\mathcal F}}
\newcommand{\ff}{\mc{F}}
\newcommand{\bff}{\ov{\mc{F}}}


\newcommand{\p}{\partial}
\newcommand{\wt}{\widetilde}
\newcommand{\ov}{\overline}
\newcommand{\mc}[1]{\mathcal{#1}}
\newcommand{\md}{\mathcal{D}}
\newcommand{\ml}{\mathcal{L}}
\newcommand{\mw}{\mathcal{W}}
\newcommand{\ma}{\mathcal{A}}


\newcommand{\GeV}{{\rm GeV}}
\newcommand{\eV}{{\rm eV}}
\newcommand{\Heff}{{\mathcal{H}_{\rm eff}}}
\newcommand{\Leff}{{\mathcal{L}_{\rm eff}}}
\newcommand{\el}{{\rm EM}}
\newcommand{\uflavor}{\mathbf{1}_{\rm flavor}}
\newcommand{\lgr}{\left\lgroup}
\newcommand{\rgr}{\right\rgroup}


\newcommand{\Mpl}{M_{\rm Pl}}
\newcommand{\suc}{{{\rm SU}_{\rm C}(3)}}
\newcommand{\sul}{{{\rm SU}_{\rm L}(2)}}
\newcommand{\sutw}{{\rm SU}(2)}
\newcommand{\suth}{{\rm SU}(3)}
\newcommand{\ue}{{\rm U}(1)}


\newcommand{\LN}{\Lambda_\text{SU($N$)}}
\newcommand{\sunu}{{\rm SU($N$) $\times$ U(1) }}
\newcommand{\sunun}{{\rm SU($N$) $\times$ U(1)}}
\def\cfl {$\text{SU($N$)}_{\rm C+F}$ }
\def\cfln {$\text{SU($N$)}_{\rm C+F}$}
\newcommand{\mUp}{m_{\rm U(1)}^{+}}
\newcommand{\mUm}{m_{\rm U(1)}^{-}}
\newcommand{\mNp}{m_\text{SU($N$)}^{+}}
\newcommand{\mNm}{m_\text{SU($N$)}^{-}}
\newcommand{\AU}{\mc{A}^{\rm U(1)}}
\newcommand{\AN}{\mc{A}^\text{SU($N$)}}
\newcommand{\aU}{a^{\rm U(1)}}
\newcommand{\aN}{a^\text{SU($N$)}}
\newcommand{\baU}{\ov{a}{}^{\rm U(1)}}
\newcommand{\baN}{\ov{a}{}^\text{SU($N$)}}
\newcommand{\lU}{\lambda^{\rm U(1)}}
\newcommand{\lN}{\lambda^\text{SU($N$)}}
\newcommand{\bxir}{\ov{\xi}{}_R}
\newcommand{\bxil}{\ov{\xi}{}_L}
\newcommand{\xir}{\xi_R}
\newcommand{\xil}{\xi_L}
\newcommand{\bzl}{\ov{\zeta}{}_L}
\newcommand{\bzr}{\ov{\zeta}{}_R}
\newcommand{\zr}{\zeta_R}
\newcommand{\zl}{\zeta_L}
\newcommand{\nbar}{\ov{n}}
\newcommand{\nnbar}{n\ov{n}}
\newcommand{\muU}{\mu_\text{U}}


\newcommand{\cpn}{CP$^{N-1}$\,}
\newcommand{\CPC}{CP($N-1$)$\times$C }
\newcommand{\CPCn}{CP($N-1$)$\times$C}


\newcommand{\lar}{\lambda_R}
\newcommand{\lal}{\lambda_L}
\newcommand{\larl}{\lambda_{R,L}}
\newcommand{\lalr}{\lambda_{L,R}}
\newcommand{\blar}{\ov{\lambda}{}_R}
\newcommand{\blal}{\ov{\lambda}{}_L}
\newcommand{\blarl}{\ov{\lambda}{}_{R,L}}
\newcommand{\blalr}{\ov{\lambda}{}_{L,R}}


\newcommand{\bgamma}{\ov{\gamma}}
\newcommand{\bpsi}{\ov{\psi}{}}
\newcommand{\bphi}{\ov{\phi}{}}
\newcommand{\bxi}{\ov{\xi}{}}


\newcommand{\qt}{\wt{q}}
\newcommand{\bq}{\ov{q}}
\newcommand{\bqt}{\overline{\widetilde{q}}}


\newcommand{\eer}{\epsilon_R}
\newcommand{\eel}{\epsilon_L}
\newcommand{\eerl}{\epsilon_{R,L}}
\newcommand{\eelr}{\epsilon_{L,R}}
\newcommand{\beer}{\ov{\epsilon}{}_R}
\newcommand{\beel}{\ov{\epsilon}{}_L}
\newcommand{\beerl}{\ov{\epsilon}{}_{R,L}}
\newcommand{\beelr}{\ov{\epsilon}{}_{L,R}}


\newcommand{\bi}{{\bar \imath}}
\newcommand{\bj}{{\bar \jmath}}
\newcommand{\bk}{{\bar k}}
\newcommand{\bl}{{\bar l}}
\newcommand{\bmm}{{\bar m}}


\newcommand{\nz}{{n^{(0)}}}
\newcommand{\no}{{n^{(1)}}}
\newcommand{\bnz}{{\ov{n}{}^{(0)}}}
\newcommand{\bno}{{\ov{n}{}^{(1)}}}
\newcommand{\Dz}{{D^{(0)}}}
\newcommand{\Do}{{D^{(1)}}}
\newcommand{\bDz}{{\ov{D}{}^{(0)}}}
\newcommand{\bDo}{{\ov{D}{}^{(1)}}}
\newcommand{\sigz}{{\sigma^{(0)}}}
\newcommand{\sigo}{{\sigma^{(1)}}}
\newcommand{\bsigz}{{\ov{\sigma}{}^{(0)}}}
\newcommand{\bsigo}{{\ov{\sigma}{}^{(1)}}}


\newcommand{\rrenz}{{r_\text{ren}^{(0)}}}
\newcommand{\bren}{{\beta_\text{ren}}}


\newcommand{\Tr}{\text{Tr}}
\newcommand{\Ts}{\text{Ts}}
\newcommand{\dm}{\hat{{\scriptstyle \Delta} m}}
\newcommand{\dmdag}{\hat{{\scriptstyle \Delta} m}{}^\dag}
\newcommand{\mhat}{\widehat{m}}
\newcommand{\deltam}{{\scriptstyle \Delta} m}
\newcommand{\nvac}{\vec{n}{}_\text{vac}}


\newcommand{\ie}{{\it i.e.}~}
\newcommand{\eg}{{\it e.g.}~}
\newcommand{\ansatz}{{\it ansatz} }


\newcommand{\ii}{\hat\imath}
\newcommand{\jj}{\hat\jmath}
\newcommand{\kk}{\hat k}


% Make the bullet list symbol a small dot
\renewcommand{\labelitemi}{{\Large$\cdot$}}


\begin{document}




%%%%%%%%%%%%%%%%%%%%%%%%%%%%%%%%%%%%%%%%%%%%%%%%%%%%%%%%%%%%%%%%%%%%%%%%%%%%%%%%
%                                                                              %
%                                                                              %
%                            T I T L E   P A G E                               %
%                                                                              %
%                                                                              %
%%%%%%%%%%%%%%%%%%%%%%%%%%%%%%%%%%%%%%%%%%%%%%%%%%%%%%%%%%%%%%%%%%%%%%%%%%%%%%%%
\begin{titlepage}


\begin{flushright}
FTPI-MINN-XX/XX, UMN-TH-XXXX/XX\\
%January 5/2010/DRAFT
\end{flushright}

\vskip 2cm
\begin{center}
{  \Large \bf  Quaternionic Wavefunction }
\end{center}
\vskip 0.5cm

\begin{center}

 {\large
 \bf   Pavel A.~Bolokhov$^{\,a}$,  Cohl Furey$^{\,b}$
 }
\end {center}

\begin{center}
$^a${\it Theoretical Physics Department, St.Petersburg State University, Ulyanovskaya~1, 
	Peterhof, St.Petersburg, 198504, Russia}\\
$^b${\it Department of Applied Mathematics and Theoretical Physics,\\
	University of Cambridge, Cambridge, CB3 OWA, UK}
\end{center}




\begin{center}
{\large\bf Abstract}
\end{center}

\hspace{0.3cm}
	We argue that quaternions form a natural language for description of the wavefunction.
	In this paper, we tap into the Dirac equation and the Schr\"odinger equation, all
	formulated in quaternionic language.
	We do not exhibit any unknown or unphysical features, as mostly was done previously.
	Instead, we demonstrate the transparency of the known features and details of the Dirac equation,
	and derive the Schr\"odinger equation as its non-relativistic limit. 
	We provide an easy to use grammar for switching between the ordinary spinor language
	and the description in terms of quaternions.
\vspace{2cm}


\end{titlepage}




%%%%%%%%%%%%%%%%%%%%%%%%%%%%%%%%%%%%%%%%%%%%%%%%%%%%%%%%%%%%%%%%%%%%%%%%%%%%%%%%
%                                                                              %
%                                                                              %
%                            I N T R O D U C T I O N                           %
%                                                                              %
%                                                                              %
%%%%%%%%%%%%%%%%%%%%%%%%%%%%%%%%%%%%%%%%%%%%%%%%%%%%%%%%%%%%%%%%%%%%%%%%%%%%%%%%
\section{Introduction}
\setcounter{equation}{0}

	The list of literature on the r\^ole of quaternions in physics is so vast that it is hardly
	possible to oversee it \cite{}.
	The majority of the literature touches base on the use of quaterions in three-dimensional
	rotations and Lorentz transformations.
	Other papers include their applications in Electrodynamics and Quantum Mechanics.
	The attractive feature of quaternions is that whenever a solution is possible in the quaternionic form,
	it is much more compact than that in the customary form of Lorentz vectors and tensors. 

	For the majority of theorists, however, quaternions are seen as a somewhat exotic subject matter,
	which neither has proven to be exceedingly effective to be adopted permanently, nor has lead
	to any new phenomena or even a new formalism.

	The other drawback, as it is perceived, is a somewhat ``strange'' mathematical language,
        often accompanied by strange results following from this language.
	While from learning Quantum Mechanics and Gauge Theories we are used to non-commutative operators,
	and non-commutative objects in general, when we see an operator of multiplication ``from the right'' $ |\, \ii $
        (an example of the so-called ``barred'' operators) it immediately induces a certain degree of skepsis.
	That is, mathematically this formalism may be interesting, but physically this seems to be
	driven away from reality and therefore unnecessary.
	Furthermore, some treatments on quaternions in Quantum Mechanics predict extra degrees of freedom for fermions,
	which we obviously do not observe.
	This is the result of the fact that quaternions are multi-component numbers,
        with a bit too many components than needed for Quantum Mechanics.
        These aspects of using quaternions in theoretical physics are enough to discourage the interest in the
        majority of theorists.

	Our goal is to cast a bridge from the regular algebraic language of physics (which is, dominantly based
	on complex numbers) to the language of quaternions.
	We argue that quaternions have always been around, and we just neglected to acknowledge them.
	There is no need or necessity for any new degrees of freedom or new physics to arise.
	We would like to present a concise dictionary, so that any theorist could connect to and appreciate
	the quaternionic formalism, which appears to be quite capacious.

	The omnipresence of quaternions is easy to observe.
	We know that Quantum Mechanics is based solely on complex numbers.
	Complex numbers provide a compact and meaningful way of both formulating and solving quantum-mechanical problems.
	With some exceptions, it would be very awkward to split the Schr\"odinger equation into its real and imanginary parts,
	and then to attempt to solve the resulting equations.
	Quantum mechanical operators of momentum, angular momentum are \emph{inherently} complex.
	That is, to say, that the wavefunction is complex too.
	
	But as soon as relativistic effects are included into Quantum Mechanics, it turns out that particles
	have \emph{spin}.
	The way to incorporate spin into the wavefunction is just to make the latter consist of two components, that is,
	to turn it into a spinor.
	The spin operator itself is then given by the Pauli matrices.
	This is where quaternions get involved.
	The algebra of Pauli matrices is the same as that of quaternionic units, loosely speaking.
	We argue that there is \emph{no inherent need} to having introduced matrices.
	Quaternionic units are all that one needs, and implicitly the wavefunction in Quantum Mechanics is quaternionic.
	A peculiar property of physics is that physical objects are actually described by \emph{complex} quaternions,
	which are also known as complexified quaternions, and not by the regular quaternions.
	This does not change the main argument, however.

	So, how does one include spin into the wavefunction by means of quaternions?
	In the literature, quaternionic wavefunction is usually introduced \emph{ad hoc},
	and the resulting Dirac or Schr\"odinger equation then includes non-physical (unobserved) degrees of freedom.
	We proceed in a very conservative way, essentially expanding on the development originally presented in [{\it thesis}]:
	we start out from the Dirac Lagrangian, carefully taking into account the spin degrees of freedom,
	and then \emph{derive} the Schr\"odinger equation.
	Spin naturally appears as a consequence of the fact that the wavefunction is quaternionic.
	Quaternionic derivations, while a bit unusual for some, are simpler than spinor derivations.
	This way we argue that quaternionic language is natural for Quantum Mechanics.

	It is not just for these reasons that we believe quaternions play a fundamental r\^ole in physics.
	There are hints that quaternions are part of the natural language for the entire Standard Model \cite{}.
	We view this work as one of the steps towards the description of the Standard Model in
	such a language.

\begin{itemize}

\item	\emph{Getting rid of matrices}

\end{itemize}


%%%%%%%%%%%%%%%%%%%%%%%%%%%%%%%%%%%%%%%%%%%%%%%%%%%%%%%%%%%%%%%%%%%%%%%%%%%%%%%%
%%%%%%%%%%%%%%%%%%%%%%%%%%%%%%%%%%%%%%%%%%%%%%%%%%%%%%%%%%%%%%%%%%%%%%%%%%%%%%%%
\vskip 0.8cm
\centerline{*\qquad\qquad\qquad*\qquad\qquad\qquad*}
\vskip 0.6cm
%%%%%%%%%%%%%%%%%%%%%%%%%%%%%%%%%%%%%%%%%%%%%%%%%%%%%%%%%%%%%%%%%%%%%%%%%%%%%%%%
%%%%%%%%%%%%%%%%%%%%%%%%%%%%%%%%%%%%%%%%%%%%%%%%%%%%%%%%%%%%%%%%%%%%%%%%%%%%%%%%


	Let us talk about our notations first, while gradually introducing the subject matter.
	The reader eager to see the physical results may choose to skip to the next section, occasinally returning here
	to clarify the notations, when necessary.
	Here we overview the known facts which are easy to pick up and use, while their proof can be found in the literature.
	We denote the quaternionic units as
\begin{equation}
	\ii^2	~~=~~	\jj^2	~~=~~	\kk^2	~~=~~	-1\,,
\end{equation}
	and we do not distinguish them from the three-dimensional unit vectors.
	That is, any three-dimensional vector is a quaternion
\begin{equation}
	\vec a	~~=~~	a^1\, \ii  ~~+~~  a^2\, \jj  ~~+~~  a^3\, \kk\,.
\end{equation}

	Complex quaternions are defined as
\begin{equation}
\label{cq}
	a	~~=~~	a^0  ~~+~~  a^1 \ii  ~~+~~  a^2 \jj  ~~+~~  a^3 \kk\,,
\end{equation}
	where all components
\begin{align}
	a^0	& ~~=~~	b^0  ~~+~~  i\,b^0\,,
	&
	a^1	& ~~=~~ b^1  ~~+~~  i\,b^1\,,
	&
	\dots,
\end{align}
	are complex numbers.
	Here $ i $ denotes a regular imaginary complex unit, which commutes with $ \ii $, $ \jj $, $ \kk $.

	We right away introduce the conjugation operations.
	We denote the quaternionic conjugation by $ \wt a $,
\begin{equation}
	\wt a	~~=~~	a^0  ~~-~~  a^1 \ii  ~~-~~  a^2 \jj  ~~-~~ a^3 \kk\,.
\end{equation}
	We remember that the quaternionic conjugation switches the order of factors in a product:
\begin{equation}
	\wt{a\, b}  ~~=~~  \wt b\, \wt a\,.
\end{equation}
	Since the components are complex numbers, we also have the complex conjugation $ a^* $
\begin{equation}
	a^*	~~=~~	(a^0)^*  ~~+~~  (a^1)^*\, \ii  ~~+~~  (a^2)^*\, \jj  ~~+~~  (a^3)^*\, \kk\,.
\end{equation}
	It is convenient to introduce the composition of these two conjugations, which we call a \emph{hermitean}
	conjugation (not without a reason),
\begin{equation}
	a^\dag	~~=~~	(a^0)^*  ~~-~~  (a^1)^*\, \ii  ~~-~~  (a^2)^*\, \jj  ~~-~~  (a^3)^*\, \kk\,.
\end{equation}
	By itself it does not give anything new, as it is merely a combination of $ \,\wt{~}\, $ and $ \,*\, $ operations,
	but it is important to have it, as we will see.
	Hermitean conjugation also interchanges the order of terms in a product, obviously
\begin{equation}
	(a\, b)^\dag	~~=~~	b^\dag\, a^\dag\,.
\end{equation}
	It is this plentitude of conjugations that give richness to the quaternionic language, when applied to physics.

	Now we can define a \emph{true} four-dimensional vector
\begin{equation}
\label{true-vector}
	v	~~=~~	v^0  ~+~  i\,(\, v^1\,\ii  ~+~  v^2\,\jj  ~+~ v^3\,\kk \,)
		~~\equiv~~	v^0  ~+~  i\,\vec v
\end{equation}
	in Minkowsky space.
	Note that it is precisely the combination $ i\,\vec v $ that gives a true vector.
	Without it,
\begin{equation}
\label{pseudo-vector}
	v	~~=~~	i\,v^0  ~+~  \vec v
\end{equation}
	describes a \emph{pseudo}-scalar and a \emph{pseudo}-vector, with respect to inversion.
	Together, the two objects \eqref{true-vector} and \eqref{pseudo-vector} span the entire space of 
	complex quaternions \eqref{cq}.
	In other words, a generic complex quaternion $ a $ can be split into two four-vectors.
	Their time components will represent a true- and a pseudo-scalar, while their three-dimensional
	parts will represent a true and an axial vector, correspondingly.
	Notice, how multiplication by the complex $ i $ turns a true vector into an axial vector, and
	the same for scalars.

	With four-dimensional vectors in Minkowski space, one has to be careful to always keep in mind whether
	a vector is contravariant or covariant.
	Complex conjugation $ * $ turns a contravariant vector $ v $ into a covariant vector $ v^* $.
\begin{equation}
	v^*	~~=~~	v^0  ~-~  i\,\vec v\,.
\end{equation}
	This actually is equivalent to quaternionic conjugation $ v^*  ~=~ v^\dag $.

	The derivative operator $ \partial $
\begin{equation}
	\partial	~~=~~	\partial^0  ~+~  i\,\vec\nabla
\end{equation}
	is \emph{by definition} a covariant vector, while obviously
\begin{equation}
	\partial^*	~~=~~	\partial^0  ~-~  i\,\vec\nabla
\end{equation}
	is a contravariant vector.

	To make these identifications more meaningful, let us talk about Lorentz transformations.
	Lorentz transformations are generated by a purely-imaginary quaterionic parameter
\begin{equation}
	\Lambda		~~=~~	\vec\kappa  ~~+~~  i\,\vec\lambda\,.
\end{equation}
	We are not going to explicitly treat it as a vector, so we are not putting a vector sign on this parameter.
	Parameter $ \vec\kappa $ generates three-dimensional rotations, while parameter $ \vec\lambda $
	generates boosts.
	These are ``generators'' in the sense that the actual finite transformations are performed by the exponent
\begin{equation}
	e^\Lambda\,.
\end{equation}
	It is important to be careful here, as $ \vec\kappa $ can be interpreted as a three-dimensional rotation
	only when $ \vec\lambda ~=~ 0 $, and the same for $ \vec\lambda $ --- it can be interpreted as a boost
	only when $ \vec\kappa ~=~ 0 $.
	This is because in general $ \vec\kappa $ and $ \vec\lambda $ do not commute, and therefore
\begin{equation}
	e^{\vec\kappa ~+~ i\,\vec\lambda}	~~\neq~~	e^{\vec\kappa}  ~\cdot~  e^{i\,\vec\lambda}\,,
\end{equation}
	where each individual exponent on the right-hand side {\emph is} treated as a rotation and a boost,
	correspondingly.
	Notice that since $ \Lambda $ is purely imaginary, $ \wt\Lambda ~=~ -\Lambda $, and therefore
\begin{equation}
	\wt{(e^\Lambda)}		~~=~~	e^{-\Lambda}\,.
\end{equation}
	A contravariant vector $ v $ transforms under $ \Lambda $ as
\begin{equation}
\label{contra}
	v	~~\to~~		e^\Lambda\,v\,e^{\Lambda^\dag}\,.
\end{equation}
	Any covariant vector then should transform the same way that $ v^* $ does:
\begin{equation}
\label{co}
	v^*	~~\to~~		e^{\Lambda^*}\,v^*\,e^{\wt\Lambda}\,.
\end{equation}
	As a special case of these, a three-dimensional vector $ \vec v $ rotates as
\begin{equation}
\label{rotation}
	\vec v	~~\to~~		e^{\vec\kappa}\,\vec v\,e^{-\vec\kappa}\,.
\end{equation}
	This concludes the basic discussion of vectors for now.

	Another way a complex quaternion \eqref{cq} can be split up, is by separating its zeroth component $ a^0 ~\equiv~ \phi $
	and its vector part $ \vec a $.
	The zeroth component $ \phi $ is identified as a complex scalar field.
	The remaining vector part is then identified as a fieldstrength:
\begin{equation}
\label{fs}
	\vec F	~~=~~	\vec B  ~+~  i\,\vec E\,.
\end{equation}
	Notice that this agrees with the identifications of \eqref{true-vector} and \eqref{pseudo-vector}
	as polar and axial vectors.
	While both $ \vec B $ and $ \vec E $ are vectors in the three-dimensional sense,
	one does not view the object \eqref{fs} as a four-dimensional vector in any way.
	This is an entirely different split up of a complex quaternion.

	Both $ \phi $ and $ \vec F $ transform the same way under Lorentz transformations:
\begin{equation}
	p	~~\to~~	e^\Lambda\,p\,e^{\wt\Lambda}\,.
\end{equation}
	For the scalar $ \phi $ this obviously does not do anything, since it is just a complex number:
\[
	e^\Lambda\,\phi\,e^{\wt\Lambda}	~~=~~	\phi\, e^\Lambda\, e^{\wt\Lambda}	~~=~~	\phi\,,
\]
	so it is indeed a scalar.
	While for the fieldstrength, this dictates that
\begin{equation}
	\vec F		~~\to~~		e^\Lambda\,\vec F\,e^{\wt\Lambda}\,.
\end{equation}
	Note that for the case of pure rotations, when $ \Lambda ~=~ \vec\kappa $,
	this agrees with Eq.~\eqref{rotation}, as $ \wt\Lambda ~=~ -\Lambda ~=~ -\vec\kappa $.
	That is, both the electric and magnetic fields rotate as three-vectors.

	Finally, we introduce spinors.
	Here we give a very brief overview of spinors necessary for Section~\ref{section-dirac}, while
	a more detailed discussion is postponed until Section~\ref{section-spinors}.
	We begin with stating that the space of complex quaternions can be split in two halves in a yet another way,
	namely using chirality projectors, $ P_L $ and $ P_R $.
	A \emph{left-handed} spinor is defined as an arbitrary complex quaternion multiplied by 
	a projector $ P_L $ on the right:
\begin{equation}
	\psi_L	~~=~~	a\,P_L\,.
\end{equation}
	Here $ P_L $ is a complex quaternion
\begin{equation}
	P_L	~~=~~	\frac{1 \,+\, i\kk} 2\,,
\end{equation}
	and we call it a projector because
\[
	P_L^2	~~=~~	P_L\,.
\]
	All accompanying details of these definitions can be found in Section~\ref{section-spinors}.
	For an ordinary reader $ \psi_L $ should be precisely viewed as a left-handed
	chiral spinor.
	A convenient basis for $ \psi_L $ is formed by elements $ P_L $ and $ \jj\,P_L $,
	which are geometrically orthogonal to each other:
\begin{equation}
\label{lbasis}
	\psi_L	~~=~~	\xi_L\,P_L  ~~+~~  \chi_L\,\jj\,P_L\,.
\end{equation}
	Complex numbers $ \xi_L $ and $ \chi_L $ are precisely the ``spin-up'' and ``spin-down''
	components of $ \psi_L $ viewed as a Weyl spinor:
\begin{equation}
	\psi_L	~~=~~	\left\lgroup
				\begin{matrix}
					\xi_L \\
					\chi_L
                		\end{matrix}
			\right\rgroup.
\end{equation}

	Left-handed spinors span a half of the complex quaternion space, while the other half
	is spanned by the right-handed spinors:
\begin{equation}
	\psi_R	~~=~~	a\,P_R\,,
\end{equation}
	where
\begin{equation}
	P_R	~~=~~	\frac{1 \,-\, i\kk} 2\,.
\end{equation}
	These two halves are related by complex conjugation $ * $,
	and the projectors are related as
\begin{align}
	P_R	& ~~=~~	P_L^*\,,
	&
	P_L  ~~+~~  P_R	& ~~=~~ 1\,.
\end{align}
	The basis for $ \psi_R $ is similarly given by $ P_R $ and $ \jj\,P_R $, but the
	components are identified slightly differently
\begin{equation}
\label{rbasis}
	\psi_R	~~=~~	-\xi_R\,\jj\,P_R  ~~+~~  \chi_R\,P_R\,.
\end{equation}
	Defined like so,
\begin{equation}
	\psi_R	~~=~~	\left\lgroup
				\begin{matrix}
					\xi_R \\
					\chi_R
                		\end{matrix}
			\right\rgroup
\end{equation}
	is precisely identified as a right-handed Weyl spinor.
	The basis \eqref{lbasis} and especially so \eqref{rbasis} may seem a bit awkward,
	but they are convenient for doing algebra.
	The fact that the projectors are part of the bases allows us to perform
	various manipulations and conjugations on spinors quite effectively.

	Under Lorentz transformations, left spinors by definition transform as
\begin{equation}
	\psi_L	~~\to~~		e^\Lambda\, \psi_L\,,
\end{equation}
	while the right-handed ones transform in the conjugate representation
\begin{equation}
	\psi_R	~~\to~~		e^{\Lambda^*}\, \psi_R\,.
\end{equation}
	It is interesting to observe and compare how spinors and three-dimensional vectors
	transform under three dimensional rotaions.
	Consider a rotation around axis $ \hat a $ ($ \hat a{}^2 ~=~ -1 $) through angle $ \alpha $.
	Vector $ \vec v $ rotates as
\begin{equation}
	\vec v	~~\to~~		e^{\alpha\, \hat a/2}\, \vec v\, e^{-\alpha\, \hat a/2}\,,
\end{equation}
	while spinors transform as
\begin{equation}
	\psi_{L,R}	~~\to~~	e^{\alpha\, \hat a/2}\, \psi_{L,R}\,.
\end{equation}
	In particular, if we perform a rotation through $ 2\pi $, then
	$ e^{\pm 2\pi\,\hat a/2} ~=~ -1 $, and a vector is unchanged,
	while a spinor changes its sign, as it should be.
	Of course, this is because of the factor of $ 1/2 $ in the exponent (the famous ``Rodrigues' two''), which
	is just the reflection of the fact that SU(2) is a double cover of
	the rotation group SO(3).

\begin{itemize}

\item 
	Parity operator acting on spinors is $ 1 | ( - \ii ) $.
	It sends $ \psi_L $ to $ i\,\psi_R $ and $ \psi_R $ to $ i\,\psi_L $.
	There are other normalizations for parity of course,
	this is just one example

\end{itemize}


%%%%%%%%%%%%%%%%%%%%%%%%%%%%%%%%%%%%%%%%%%%%%%%%%%%%%%%%%%%%%%%%%%%%%%%%%%%%%%%%
%                                                                              %
%                                                                              %
%                          D I R A C   E Q U A T I O N                         %
%                                                                              %
%                                                                              %
%%%%%%%%%%%%%%%%%%%%%%%%%%%%%%%%%%%%%%%%%%%%%%%%%%%%%%%%%%%%%%%%%%%%%%%%%%%%%%%%
\section{Dirac equation}
\label{section-dirac}
\setcounter{equation}{0}

	Our goal in this section is to construct the Lagrangian for the electron in electromagnetic field,
	in quaternionic form, and derive the equation of motion --- the Dirac equation.

	Let us begin with a massless particle, in the absence of the electromagnetic field.
	The appropriate Lagrangian was given in [{\it thesis}],
\begin{equation}
\label{L-massless}
	\mc L_\text{massless}
		~~=~~	\psi_L^\dag\, i\partial\, \psi_L  ~~+~~  \psi_R^\dag\, i\,\partial^*\, \psi_R
			~~+~~  \text{c.c.}
\end{equation}
	Note that Lorentz invariance is manifest here, because
\begin{align}
	\psi_L^\dag	& ~~\to~~	\psi_L^\dag\, e^{\Lambda^\dag}  ~~=~~  \psi_L^\dag\, e^{-\Lambda^*}\,,
	&
	\partial	& ~~\to~~	e^{\Lambda^*}\, \partial\, e^{-\Lambda}\,,
	&
	\psi_L		& ~~\to~~	e^\Lambda\, \psi_L\,,
\end{align}
	and
\begin{align}
	\psi_R^\dag	& ~~\to~~	\psi_R^\dag\, e^{\wt \Lambda}  ~~=~~  \psi_R^\dag\, e^{-\Lambda}\,,
	&
	\partial^*	& ~~\to~~	e^\Lambda\, \partial^*\, e^{-\Lambda^*}\,,
	&
	\psi_R		& ~~\to~~	e^{\Lambda^*}\, \psi_R\,.
\end{align}
{\boldmath\bf
	Although both $ \psi_L $ and $ \psi_R $ are parts of a single Dirac spinor
\begin{equation}
	\psi_D		~~=~~	\psi_L  ~~+~~  \psi_R\,,
\end{equation}
	it does not seem to be possible to combine the two terms in Eq.~\eqref{L-massless}
	into a single term just using $ \psi_D $,
	because of the different representations of the derivative $ \partial $
	---
	should think about this.
}

	We need to make an important remark about conjugating products of spinors,
	due to the fact that spinors are Grasmann variables.
	By definition, complex conjugation of two Grassman variable interchanges their order,
\begin{equation}
	(\zeta\, \eta)^*	~~=~~	\eta^*\, \zeta^*\,,\qquad\qquad \text{for Grassmann numbers.}
\end{equation}
	If we take two complex quaternions $ \xi $ and $ \chi $, which are fermions at the same time,
	complex conjugation cannot change their order, because of their quaternionic content.
	In that case, the order is preserved, but an extra minus sign appears,
\begin{equation}
	(\xi\, \chi)^*		~~=~~	-\, \xi^*\, \chi^*\,,\qquad \text{for fermionic complex quaternions.}
\end{equation}
	If we take a quaternionic conjugate $ \wt{\xi\, \chi} $, on the other hand, 
	the conjugation will attempt to change their order precisely because of the quaternionic content.
	Note that the quaternionic algebra \emph{requires} us to interchange the factors, or
	the result will simply be incorrect.
	But now because the spinors are fermions, and we are \emph{not} performing a
	complex conjugation, we get an extra minus sign
\begin{equation}
	\wt{\xi\, \chi}		~~=~~	-\, \wt\chi\, \wt\xi\,\qquad \text{for quaternionic spinors.}
\end{equation}
	The only kind of conjugation which does not produce a negative sign is hermitean
	conjugation --- this combination changes the order of the spinors in agreement with
	both complex and quaternionic conjugations,
\begin{equation}
	(\xi\, \chi)^\dag	~~=~~	\chi^\dag\, \xi^\dag\,.
\end{equation}

	Let us discuss gauge transformations now, because we need the electron to interact with the electromagnetic field.
	Gauge transformations just rotate the overall complex phase of a spinor, and so they are defined
	similarly both for right- and left-handed spinors,
\begin{align}
	\psi_L		& ~~\to~~		e^{i\varphi}\,\psi_L\,,
	&
	\psi_R		& ~~\to~~		e^{i\varphi}\,\psi_R\,.
\end{align}
	Notice that the quaternionic conjugates also transform the same way,
\begin{align}
	\wt\psi{}_L		& ~~\to~~		\wt\psi{}_L\,e^{i\varphi}\,,
	&
	\wt\psi{}_R		& ~~\to~~		\wt\psi{}_R\,e^{i\varphi}\,.
\end{align}
	Although $ e^{i\varphi} $ certainly commutes with $ \wt\psi{}_{L,R} $,
	for convenience we wrote it on the right of the latter.
	Both the complex conjugates and hermitean conjugates will have the opposite charge,
\begin{align}
%
\notag
	\psi_L^*		& ~~\to~~		\psi_L^*\,e^{-i\varphi}\,,
	&
	\psi_R^*		& ~~\to~~		\psi_R^*\,e^{-i\varphi}\,,
	\\[4mm]
%
	\psi_L^\dag		& ~~\to~~		\psi_L^\dag\,e^{-i\varphi}\,,
	&
	\psi_R^\dag		& ~~\to~~		\psi_R^\dag\,e^{-i\varphi}\,.
\end{align}

	In order to make this transformation local, we define the long derivative,
\begin{equation}
	\md		~~=~~	\partial  ~~-~~  i\,A^*\,.
\end{equation}
	The reason that we have to put $ A^* $ here instead of just $ A $ is because
	$ \partial $ and $ \md $ are covariant vectors.
	This is just the reflection of the fact that $ A^\mu $ enters the long derivative
	with the lower index $ \mu $:
\begin{equation}
	\md_\mu		~~=~~	\partial_\mu  ~~-~~  i\,A_\mu\,.
\end{equation}
	This long derivative then transforms as
\begin{equation}
	\md		~~\to~~		e^{i\varphi}\, \md\, e^{-i\varphi}\,,
\end{equation}
	meaning that, as usual,
\begin{equation}
	A_\mu		~~\to~~		A_\mu  ~~+~~  \partial_\mu\,\varphi\,.
\end{equation}
	This allows $ \md $ to act on $ \psi_L $, so that $ \md\,\psi_L $ is again in the fundamental representation
	of U(1).
	Now, although $ \psi_R $ has the same charge as $ \psi_L $, we cannot act on it with the same derivative, because
	the product $ \md\,\psi_R $ will not transform under Lorentz transformations properly.
	Remind, that the right-handed and left-handed spinor spaces are in fact related by complex conjugation.
	This was the reason that we wrote $ \partial^* $ in the Lagrangian in Eq.~\eqref{L-massless}.
	One would think that by analogy we should act on $ \psi_R $ with $ \md^* $ --- but that would also be
	a mistake because it would imply that $ \psi_R $ has the opposite charge.
	In reality, it is the quaternionic conjugate $ \wt\md $ that should be put into the Lagrangian.
	This conjugation does not change the sign of the electric charge.
	Overall, the Lagrangian now looks as,
\begin{equation}
\label{L-gauge}
	\mc L_\text{massless}
		~~=~~	\psi_L^\dag\, i\md\, \psi_L  ~~+~~  \psi_R^\dag\, i\,\wt\md\, \psi_R
			~~+~~  \text{c.c.}
\end{equation}
	Before proceeding, let us emphasize the remarkable feature of the Lagrangians \eqref{L-massless} and \eqref{L-gauge},
	which we could have done earlier: the absence of $ \gamma $-matrices (or $ \sigma $-matrices for that matter).
	The only residue of the matrix structure of the Dirac's Lagrangian is residing in the fact that the Lagragians
	have two chiral terms, instead of just one.
	We will return to this below.

	Now we add the mass term.
	Since this has to be the Dirac mass, it has to flip chirality.
	The only form that correctly reproduces the mass term is $ m\, \jj\, \psi_L^\dag\, \psi_R $,
\begin{equation}
	m\, \jj\, \psi_L^\dag\, \psi_R  ~~+~~  \text{q.c.}  ~~+~~  \text{c.c}  ~~=~~
	m \lgr \jj\, \psi_L^\dag\, \psi_R  ~+~  \wt\psi{}_R\, \psi_L^*\, \jj \rgr
	~~+~~  \text{c.c.}
\end{equation}
	We will discuss the occurrence of $ \jj $ in this expression in detail in Section~\ref{section-spinors}.
	Its appearance seems ugly, and we will be able to get rid of it soon, after we discover its meaning.
	For now we just note that without it the expression would vanish, \emph{i.e.} $ \psi_L^\dag\, \psi_R $
	has no real part.
	Let us also note that this factor of $ \jj $ can be moved to the right at our conveniece:
\begin{equation}
\label{m-term}
	m\, \psi_L^\dag\, \psi_R\, \jj  ~~+~~  \text{q.c.}  ~~+~~  \text{c.c.}
\end{equation}
	This follows from the general cyclic property of quaternion product
	under the ``$ \text{q.c.} $'' sign, analogous to the cyclicity of trace of matrices\footnote{
		In fact, if one chooses to represent quaternions via Pauli matrices, the real
		part of a quaternion exactly corresponds to the trace of its matrix representation.
	} ---
\begin{equation}
	a\, b\, c  ~~+~~  \text{q.c.}	~~=~~	b\, c\, a  ~~+~~  \text{q.c.}
\end{equation}
	The proof is simple --- the real part of a product of \emph{two} quaternions cannot depend
	on their order.
	From this follows the cyclicity.
	The fact that we are dealing with complexified quaternions cannot change this property.

	As we will see in Section~\ref{section-spinors}, in the product $ \psi_R\, \jj $ the factor $ \jj $
	``elevates'' the right-handed spinor $ \psi_R $ to the left-handed space
	(we have used the term ``elevates'' because conventionally a left-handed spinor is
	written above right-handed spinor inside the column of a Dirac spinor).
	Importantly, the factor $ \jj $ does not change the spinor's representation (it is obviously still transformed
	via multiplication by $ e^{\Lambda^*} $ on the left).
	Instead, the spinor just becomes expandable in the left-handed basis \eqref{lbasis}.
	We will use this when we define the \emph{standard} representation for spinors below.

	Written in terms of the components, the mass term \eqref{m-term} gives
\begin{equation}
	\psi_L^\dag\, \psi_R\, \jj  ~~+~~  \text{q.c.}  ~~+~~  \text{c.c.}	~~=~~
	\xi_L^*\,\xi_R  ~+~ \chi_L^*\,\chi_R  ~+~  \xi_R^*\,\xi_L  ~+~ \chi_R^*\,\chi_L\,,
\end{equation}
	as it should be for the Dirac mass term.


%%%%%%%%%%%%%%%%%%%%%%%%%%%%%%%%%%%%%%%%%%%%%%%%%%%%%%%%%%%%%%%%%%%%%%%%%%%%%%%%
%                                                                              %
%                        D I R A C   L A G R A N G I A N                       %
%                                                                              %
%%%%%%%%%%%%%%%%%%%%%%%%%%%%%%%%%%%%%%%%%%%%%%%%%%%%%%%%%%%%%%%%%%%%%%%%%%%%%%%%
\subsection{Dirac Lagrangian}
	Now we can derive the Dirac equation.
	Our starting point is the full Lagrangian
\begin{equation}
\label{L-massive}
	\ml_\text{Dirac}	~~=~~	
			\psi_L^\dag\, i\md\, \psi_L  ~~+~~  \psi_R^\dag\, i\,\wt\md\, \psi_R
			~~-~~  
			\lgr m\, \psi_L^\dag\, \psi_R\, \jj  ~~+~~  \text{q.c.} \rgr  ~~+~~  \text{c.c.},
\end{equation}
	where we remember that the mass term has to actually enter with a negative sign,
	and we consider the electromagnetic field to be fixed.

	To derive the equations of motion, we vary the Lagrangian \eqref{L-massive} with respect to
	$ \psi_L^\dag $ and $ \psi_R^\dag $.
	The reader may wonder at this point -- how are we going to differentiate this Lagrangian with respect
	to quaternions, let alone complexified and fermionic?
	We postpone the formal answer to this question until Section~\ref{section-diff}.
	For now, we can just act intuitively, at least when differentiating with respect to $ \psi_L^\dag $.
	Indeed, in the terms where it is present in Eq.~\eqref{L-massive}, it is sitting on the left,
	and so the left derivative gives
\begin{equation}
\label{ldirac}
	i\,\md\,\psi_L  ~~-~~  m\, \psi_R\, \jj	~~=~~	0\,.
\end{equation}
	Loosely speaking, for quaternions $ \partial \wt q/\partial q ~\neq~ 0 $ --- unlike for complex numbers,
	for which $ \partial \ov z / \partial z ~=~ 0 $.
	So it seems like there should be more terms on the left-hand side of Eq.~\eqref{ldirac}.
	Why we can act so naively, and why the other terms do not contribute is, again,
	explained in Section~\ref{section-diff}.
	Here we provide an alternative and a more transparent justification, as follows.
	Let us for a moment pretend that we are dealing with a ``quaternionic'' Lagrangian
\begin{equation}
\label{L-quat}
	\psi_L^\dag\, i\md\, \psi_L  ~~+~~  m\, \psi_L^\dag\, \psi_R\, \jj
\end{equation}
	of which we will only want its real part.
	If we find the extremum of this Lagrangian, it will also extremize its real part.
	But the extremum of \eqref{L-quat} is exactly given by Eq.~\eqref{ldirac}.

	In order to vary with respect to $ \psi_R^\dag $, we just re-write the mass term as
\begin{equation}
	\ml_\text{Dirac}		~~\supset~~
			\lgr m\, \psi_R^\dag\, \psi_L\, \jj  ~~+~~  \text{q.c.} \rgr  ~~+~~  \text{c.c.}
\end{equation}
	We did not make anything up here, as the first term here
	was just hidden inside the ``$ \text{q.c.} $'' and ``$ \text{c.c.} $'' in Eq.~\eqref{L-massive}.
	Notice that this term enters with a \emph{positive} sign.

	Now we can vary the Langrangian with respect to $ \psi_R^\dag $, finding
\begin{equation}
\label{rdirac}
	i\,\wt\md\,\psi_R  ~~+~~  m\, \psi_L\, \jj	~~=~~	0\,.
\end{equation}
	The two expressions \eqref{ldirac} and \eqref{rdirac} are the \emph{quaternionic Dirac equations}.


%%%%%%%%%%%%%%%%%%%%%%%%%%%%%%%%%%%%%%%%%%%%%%%%%%%%%%%%%%%%%%%%%%%%%%%%%%%%%%%%
%                                                                              %
%                  N O N - R E L A T I V I S T I C   L I M I T                 %
%                                                                              %
%%%%%%%%%%%%%%%%%%%%%%%%%%%%%%%%%%%%%%%%%%%%%%%%%%%%%%%%%%%%%%%%%%%%%%%%%%%%%%%%
\subsection{Non-relativistic limit}


\pagebreak
\begin{itemize}

\item Gauge transformations

\item Dirac lagrangian

\item Standard representation

\item Dirac equation

\item Non-relativistic limit

\item Current

\end{itemize}




%%%%%%%%%%%%%%%%%%%%%%%%%%%%%%%%%%%%%%%%%%%%%%%%%%%%%%%%%%%%%%%%%%%%%%%%%%%%%%%%
%                                                                              %
%                                                                              %
%                                 S P I N O R S                                %
%                                                                              %
%                                                                              %
%%%%%%%%%%%%%%%%%%%%%%%%%%%%%%%%%%%%%%%%%%%%%%%%%%%%%%%%%%%%%%%%%%%%%%%%%%%%%%%%
\section{Spinors}
\label{section-spinors}




%%%%%%%%%%%%%%%%%%%%%%%%%%%%%%%%%%%%%%%%%%%%%%%%%%%%%%%%%%%%%%%%%%%%%%%%%%%%%%%%
%                                                                              %
%                                                                              %
%                         D I F F E R E N T I A T I O N                        %
%                                                                              %
%                                                                              %
%%%%%%%%%%%%%%%%%%%%%%%%%%%%%%%%%%%%%%%%%%%%%%%%%%%%%%%%%%%%%%%%%%%%%%%%%%%%%%%%
\section{Differentiation}
\label{section-diff}




%%%%%%%%%%%%%%%%%%%%%%%%%%%%%%%%%%%%%%%%%%%%%%%%%%%%%%%%%%%%%%%%%%%%%%%%%%%%%%%%
%                                                                              %
%                                                                              %
%                             C O N C L U S I O N S                            %
%                                                                              %
%                                                                              %
%%%%%%%%%%%%%%%%%%%%%%%%%%%%%%%%%%%%%%%%%%%%%%%%%%%%%%%%%%%%%%%%%%%%%%%%%%%%%%%%
\section{Conclusions}




%%%%%%%%%%%%%%%%%%%%%%%%%%%%%%%%%%%%%%%%%%%%%%%%%%%%%%%%%%%%%%%%%%%%%%%%%%%%%%%%
%                                                                              %
%                                                                              %
%                          A C K N O W L E D G M E N T S                       %
%                                                                              %
%                                                                              %
%%%%%%%%%%%%%%%%%%%%%%%%%%%%%%%%%%%%%%%%%%%%%%%%%%%%%%%%%%%%%%%%%%%%%%%%%%%%%%%%
\section*{Acknowledgments}

\small
\begin{thebibliography}{99}
\itemsep -2pt

  \bibitem{HT1}
  A.~Hanany and D.~Tong,
  %``Vortices, instantons and branes,''
  JHEP {\bf 0307}, 037 (2003)
  [hep-th/0306150].
  %%CITATION = HEP-TH 0306150;%%

  \bibitem{SYmon}
  M.~Shifman and A.~Yung,
  %``Non-Abelian string junctions as confined monopoles,''
  Phys.\ Rev.\ D {\bf 70}, 045004 (2004)
  [hep-th/0403149].
  %%CITATION = HEP-TH 0403149;%%

  \bibitem{HT2}
  A.~Hanany and D.~Tong,
  %``Vortex strings and four-dimensional gauge dynamics,''
  JHEP {\bf 0404}, 066 (2004)
  [hep-th/0403158].
  %%CITATION = HEP-TH 0403158;%%

  \bibitem{Trev}
  D.~Tong,
  %``Quantum Vortex Strings: A Review,''
  Annals Phys.\  {\bf 324}, 30 (2009)
  [arXiv:0809.5060 [hep-th]];
  %%CITATION = APNYA,324,30;%%
  M.~Eto, Y.~Isozumi, M.~Nitta, K.~Ohashi and N.~Sakai,
  %``Solitons in the Higgs phase: The moduli matrix approach,''
  J.\ Phys.\ A  {\bf 39}, R315 (2006)
  [arXiv:hep-th/0602170];
  %%CITATION = JPAGB,A39,R315;%%
  K.~Konishi,
  %``The magnetic monopoles seventy-five years later,''
  Lect.\ Notes Phys.\  {\bf 737}, 471 (2008)
  [arXiv:hep-th/0702102];
  %%CITATION = LNPHA,737,471;%%
  M.~Shifman and A.~Yung,
  {\sl Supersymmetric Solitons,}
  (Cambridge University Press, 2009).
  %%CITATION = HEP-TH/0703267;%%

  \bibitem{W79}
  E.~Witten,
  %``Instantons, The Quark Model, And The 1/N Expansion,''
  Nucl.\ Phys.\ B {\bf 149}, 285 (1979).
  %%CITATION = NUPHA,B149,285;%%

  \bibitem{W93}
  E.~Witten,
  %``Phases of N = 2 theories in two dimensions,''
  Nucl.\ Phys.\ B {\bf 403}, 159 (1993)
  [hep-th/9301042].
  %%CITATION = HEP-TH 9301042;%%
  
  \bibitem{dadvl}
  A.~D'Adda, P.~Di Vecchia and M.~L\"{u}scher,
  %``Confinement And Chiral Symmetry Breaking In Cp**N-1 Models With Quarks,''
  Nucl.\ Phys.\  B {\bf 152}, 125 (1979).
  %%CITATION = NUPHA,B152,125;%%
  
  \bibitem{SYhet}
  M.~Shifman and A.~Yung,
  %``Large-N Solution of the Heterotic N=(0,2) Two-Dimensional CP(N-1) Model,''
  Phys.\ Rev.\  D {\bf 77}, 125017 (2008)
  [arXiv:0803.0698 [hep-th]].
  %%CITATION = PHRVA,D77,125017;%%
  
  \bibitem{MR1}
  K.~Hori and C.~Vafa,
  {\em Mirror symmetry,}
  arXiv:hep-th/0002222.
  %%CITATION = HEP-TH/0002222;%%
  
  \bibitem{MR2}
  E.~Frenkel and A.~Losev,
  %``Mirror symmetry in two steps: A-I-B,''
  Commun.\ Math.\ Phys.\  {\bf 269}, 39 (2006)
  [arXiv:hep-th/0505131].
  %%CITATION = CMPHA,269,39;%%
  
  \bibitem{EdTo}
  M.~Edalati and D.~Tong,
  %``Heterotic vortex strings,''
  JHEP {\bf 0705}, 005 (2007)
  [arXiv:hep-th/0703045].
  %%CITATION = JHEPA,0705,005;%%
  
\bibitem{SY1}
  M.~Shifman and A.~Yung,
  %``Heterotic Flux Tubes in N=2 SQCD with N=1 Preserving Deformations,''
  Phys.\ Rev.\  D {\bf 77}, 125016 (2008)
  [arXiv:0803.0158 [hep-th]].
  %%CITATION = PHRVA,D77,125016;%%
  
  \bibitem{BSY1}
  P.~A.~Bolokhov, M.~Shifman and A.~Yung,
  %``Description of the Heterotic String Solutions in U(N) SQCD,''
  Phys. \ Rev. \ D {\bf 79}, 085015 (2009) (Erratum: Phys. Rev. D 80, 049902 (2009))
  [arXiv:0901.4603 [hep-th]].
  %%CITATION = ARXIV:0901.4603;%%
  
  \bibitem{BSY2}
  P.~A.~Bolokhov, M.~Shifman and A.~Yung,
  %``Description of the Heterotic String Solutions in the M Model,''
  Phys. \ Rev. \ D {\bf 79}, 106001 (2009) (Erratum: Phys. Rev. D 80, 049903 (2009))
  [arXiv:0903.1089 [hep-th]].
  %%CITATION = ARXIV:0903.1089;%%  
  
  \bibitem{BSY3}
  P.~A.~Bolokhov, M.~Shifman and A.~Yung,
  %``Heterotic N=(0,2) CP(N-1) Model with Twisted Masses,''
  Phys.\ Rev.\  D {\bf 81}, 065025 (2010)
  [arXiv:0907.2715 [hep-th]].
  %%CITATION = PHRVA,D81,065025;%%
  
  \bibitem{orco}
  E.~Witten,
  %``A Supersymmetric Form Of The Nonlinear Sigma Model In Two-Dimensions,''
  Phys.\ Rev.\  D {\bf 16}, 2991 (1977);
  %%CITATION = PHRVA,D16,2991;%%
  P.~Di Vecchia and S.~Ferrara,
  %``Classical Solutions In Two-Dimensional Supersymmetric Field Theories,''
  Nucl.\ Phys.\  B {\bf 130}, 93 (1977).
  %%CITATION = NUPHA,B130,93;%%

  \bibitem{Bruno}
  B.~Zumino,
  %``Supersymmetry And Kahler Manifolds,''
  Phys.\ Lett.\  B {\bf 87}, 203 (1979).
  %%CITATION = PHLTA,B87,203;%%

  \bibitem{rev1}
  V.~A.~Novikov, M.~A.~Shifman, A.~I.~Vainshtein and V.~I.~Zakharov,
  %``Two-Dimensional Sigma Models: Modeling Nonperturbative Effects Of Quantum
  %Chromodynamics,''
  Phys.\ Rept.\  {\bf 116}, 103 (1984).
  %%CITATION = PRPLC,116,103;%%
   
  \bibitem{rev2} 
  A.~M.~Perelomov,
  %``SUPERSYMMETRIC CHIRAL MODELS: GEOMETRICAL ASPECTS,''
  Phys.\ Rept.\  {\bf 174}, 229 (1989).
  %%CITATION = PRPLC,174,229;%%
  
  \bibitem{WI}
  E.~Witten,
  %``Constraints On Supersymmetry Breaking,''
  Nucl.\ Phys.\  B {\bf 202}, 253 (1982).
  %%CITATION = NUPHA,B202,253;%%
  
  \bibitem{twisted}
  L.~Alvarez-Gaum\'{e} and D.~Z.~Freedman,
  %``Potentials For The Supersymmetric Nonlinear Sigma Model,''
  Commun.\ Math.\ Phys.\  {\bf 91}, 87 (1983);
  %%CITATION = CMPHA,91,87;%%
  S.~J.~Gates,
  %``Superspace Formulation Of New Nonlinear Sigma Models,''
  Nucl.\ Phys.\ B {\bf 238}, 349 (1984);
  %%CITATION = NUPHA,B238,349;%%
  S.~J.~Gates, C.~M.~Hull and M.~Ro\v{c}ek,
  %``Twisted Multiplets And New Supersymmetric Nonlinear Sigma Models,''
  Nucl.\ Phys.\ B {\bf 248}, 157 (1984).
  %%CITATION = NUPHA,B248,157;%%

  \bibitem{BelPo}
  A.~M.~Polyakov,
  %``Interaction Of Goldstone Particles In Two-Dimensions. Applications To
  %Ferromagnets And Massive Yang-Mills Fields,''
  Phys.\ Lett.\  B {\bf 59}, 79 (1975).
  %%CITATION = PHLTA,B59,79;%%
  
  \bibitem{adam}
  A.~Ritz, M.~Shifman and A.~Vainshtein,
  %``Counting domain walls in N = 1 super Yang-Mills,''
  Phys.\ Rev.\  D {\bf 66}, 065015 (2002)
  [arXiv:hep-th/0205083].
  %%CITATION = PHRVA,D66,065015;%%
  
  \bibitem{WessBagger}
  J. Wess and J. Bagger, {\em Supersymmetry and Supergravity}, Second Edition,
  Princeton University Press, 1992.

  \bibitem{Helgason}
  S. Helgason, {\sl Differential geometry, Lie groups and symmetric spaces},
  Academic Press, New York, 1978.
  
  \bibitem{Dor}
  N.~Dorey,
  %``The BPS spectra of two-dimensional
  %supersymmetric gauge theories
  %with  twisted mass terms,''
  JHEP {\bf 9811}, 005 (1998) [hep-th/9806056].
  %%CITATION = HEP-TH 9806056;%%

  \bibitem{Witten:2005px}
  E.~Witten,
  {\em Two-dimensional models with (0,2) supersymmetry: Perturbative aspects,}
  arXiv:hep-th/0504078.
  %%CITATION = HEP-TH/0504078;%%
  
  \bibitem{GSYphtr}
  A.~Gorsky, M.~Shifman and A.~Yung,
   %``Higgs and Coulomb/confining phases in "twisted-mass" deformed \cpn model,''
  Phys.\ Rev.\ D {\bf 73}, 065011 (2006)
  [hep-th/0512153].

  \bibitem{Coleman}
  S.~R.~Coleman,
  %``More About The Massive Schwinger Model,''
  Annals Phys.  {\bf 101}, 239 (1976).

  \bibitem{GSY05}
  A.~Gorsky, M.~Shifman and A.~Yung,
   %``Non-Abelian Meissner effect in Yang-Mills theories at weak
  %coupling,''
  Phys.\ Rev.\ D {\bf 71}, 045010 (2005)
  [hep-th/0412082].
  %%CITATION = HEP-TH 0412082;%%

  \bibitem{Ferrari}
  F.~Ferrari,
  % ``LARGE N AND DOUBLE SCALING LIMITS IN TWO-DIMENSIONS.''
  JHEP {\bf 0205} 044 (2002)
  [hep-th/0202002].

  \bibitem{Ferrari2}
  F.~Ferrari,
  %'' NONSUPERSYMMETRIC COUSINS OF SUPERSYMMETRIC GAUGE THEORIES:
  %QUANTUM SPACE OF PARAMETERS AND DOUBLE SCALING LIMITS.''
   Phys. Lett. {\bf B496} 212 (2000)
  [hep-th/0003142];
  %``A model for gauge theories with Higgs fields,''
  JHEP {\bf 0106}, 057 (2001)
  [hep-th/0102041].
  %%CITATION = HEP-TH 0102041;%%

  \bibitem{AdDVecSal}
  A.~D'Adda, A.~C.~Davis, P.~DiVeccia and P.~Salamonson,
  %"An effective action for the supersymmetric CP$^{n-1}$ models,"
  Nucl.\ Phys.\ {\bf B222} 45 (1983).

  \bibitem{ChVa}
  S.~Cecotti and C. Vafa,
  %"On classification of \ntwo supersymmetric theories,"
  Comm. \ Math. \ Phys. \ {\bf 158} 569 (1993)
  [hep-th/9211097].

  \bibitem{HaHo}
  A.~Hanany and K.~Hori,
  %``Branes and N = 2 theories in two dimensions,''
  Nucl.\ Phys.\  B {\bf 513}, 119 (1998)
  [arXiv:hep-th/9707192].
  %%CITATION = NUPHA,B513,119;%%

  \bibitem{AD}
  P. C.~Argyres and M. R.~Douglas,
  %``New Phenomena in SU(3) Supersymmetric Gauge Theory'' 
  Nucl. \ Phys. \ {\bf B448}, 93 (1995)   
  [arXiv:hep-th/9505062].
  %%CITATION = NUPHA,B448,93;%%
  
  \bibitem{APSW}
  P. C. Argyres, M. R. Plesser, N. Seiberg, and E. Witten,
  %``New N=2 Superconformal Field Theories in Four Dimensions''
  Nucl. \ Phys.  \ {\bf B461}, 71 (1996) 
  [arXiv:hep-th/9511154].
  %%CITATION = NUPHA,B461,71;%%

  \bibitem{SYrev}
  M.~Shifman and A.~Yung,
  %{\sl Supersymmetric Solitons,}
  Rev.\ Mod.\ Phys. {\bf 79} 1139 (2007)
  [arXiv:hep-th/0703267].
  %%CITATION = HEP-TH/0703267;%%

  \bibitem{Tonghetdyn}
  D.~Tong,
  %``The quantum dynamics of heterotic vortex strings,''
  JHEP {\bf 0709}, 022 (2007)
  [arXiv:hep-th/0703235].
  %%CITATION = JHEPA,0709,022;%%
  
  \bibitem{VYan}
  G.~Veneziano and S.~Yankielowicz,
  %``An Effective Lagrangian For The Pure N=1 Supersymmetric Yang-Mills
  %Theory,''
  Phys.\ Lett.\  B {\bf 113}, 231 (1982).
  %%CITATION = PHLTA,B113,231;%%
  
  \bibitem{ls}
  A.~Losev and M.~Shifman,
  %``N = 2 sigma model with twisted mass and superpotential: Central charges
  %and solitons,''
  Phys.\ Rev.\  D {\bf 68}, 045006 (2003)
  [arXiv:hep-th/0304003].
  %%CITATION = PHRVA,D68,045006;%%
  
  \bibitem{ls1}
  M.~Shifman, A.~Vainshtein and R.~Zwicky,
  %``Central charge anomalies in 2D sigma models with twisted mass,''
  J.\ Phys.\ A  {\bf 39}, 13005 (2006)
  [arXiv:hep-th/0602004].
  %%CITATION = JPAGB,A39,13005;%
  
  \bibitem{SYneww}
  M.~Shifman and A.~Yung,
  %``N=(0,2) Deformation of the N=(2,2) Wess-Zumino Model in Two Dimensions,''
  Phys.\ Rev.\  D {\bf 81}, 105022 (2010)
  [arXiv:0912.3836 [hep-th]].
  %%CITATION = PHRVA,D81,105022;%%
  
  \bibitem{D1}
  J.~Distler and S.~Kachru,
  %``(0,2) Landau-Ginzburg theory,''
  Nucl.\ Phys.\  B {\bf 413}, 213 (1994)
  [arXiv:hep-th/9309110].
  %%CITATION = NUPHA,B413,213;%%
  
  \bibitem{D2}
  T.~Kawai and K.~Mohri,
  %``Geometry Of (0,2) Landau-Ginzburg Orbifolds,''
  Nucl.\ Phys.\  B {\bf 425}, 191 (1994)
  [arXiv:hep-th/9402148].
  %%CITATION = NUPHA,B425,191;%%
  
  \bibitem{D3}
  I.~V.~Melnikov,
  %``(0,2) Landau-Ginzburg Models and Residues,''
  JHEP {\bf 0909}, 118 (2009)
  [arXiv:0902.3908 [hep-th]].
  %%CITATION = JHEPA,0909,118;%%
  
  \bibitem{Shifman:2009ay}
  M.~Shifman and A.~Yung,
  %``Crossover between Abelian and non-Abelian confinement in N=2 supersymmetric
  %QCD,''
  Phys.\ Rev.\  D {\bf 79}, 105006 (2009)
  [arXiv:0901.4144 [hep-th]].
  %%CITATION = PHRVA,D79,105006;%%

  \bibitem{Tadpoint}
  D.~Tong,
  %``Superconformal  vortex strings,''
  JHEP {\bf 0612}, 051 (2006)
  [arXiv:hep-th/0610214].

  \bibitem{SMMS}
  A.~Migdal and M.~Shifman,
  %``Dilaton Effective Lagrangian In Gluodynamics,''
  Phys.\ Lett.\  B {\bf 114}, 445 (1982).
  %%CITATION = PHLTA,B114,445;%%
  
  \bibitem{Kos}
  A.~Kovner and M.~A.~Shifman,
  %``Chirally symmetric phase of supersymmetric gluodynamics,''
  Phys.\ Rev.\  D {\bf 56}, 2396 (1997)
  [arXiv:hep-th/9702174].
  %%CITATION = PHRVA,D56,2396;%%

\end{thebibliography}


\end{document}
